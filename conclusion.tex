\section{結論}

% IME はアプリケーションと独立に実装されているため、
% 現在のような実装ではアプリケーションの内部状態によって動作を変えたり、
% アプリケーションの振る舞いを制御したりすることはできず、表に出ているテキストの編集操作しか
% できない。 既存のシステム自体は変更せずに、皮をかぶせる形で機能を拡張す
% る手法はある程度有用ではあるが、問題の根本的な解決が必要な場合には限界
% がある。テキスト編集の場合は根本的に解決しなければならない問題は多くな
% いので、本論文の手法はとりあえず有効だといえるが、根本的な解決のために
% は、テキスト入力の枠組みであるIMEに加えて、各コンピュータがテキスト編集
% のための枠組みを用意する必要があるだろう。
