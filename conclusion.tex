\section{結論}


OSに標準装備されたIME機能を活用することにより,
様々なテキストエディタにおける入力/編集操作を共通化する手法を提案した.
IME機能はアプリケーションから独立しているため,
アプリケーション内のデータを操作することはできないが,
様々なアプリケーションにおける
入力/編集操作の多くの部分を共通化することができた.
%

テキストエディタの研究は長い歴史を持っているが\cite{texteditors.org},
近年はタブレットやモバイル環境における入力手法に関する研究%
\cite{Li:1lineKB}%
\cite{MacKenzie:H4Writer}%
\cite{Rick:VirtualKB}%
や特殊装置を使う文字入力装置に関する研究%
\cite{Dietz:PressureKB}%
\cite{Harrison:Skinput}%
\cite{Murase:CameraKB}%
\cite{Wigdor:TiltKB}
が主流になっており,一般的なテキストエディタを改善する研究は少ないようである.
テキストの入力や編集は計算機利用における最も重要な仕事のひとつであることは
間違いないので,
操作を簡単化/共通化するための優れた枠組みの開発は重要な課題だと考えられる.

