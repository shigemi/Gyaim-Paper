\section{実装}

{\system} は MacRuby で記述された Mac 用の IME であり,ソースが 500 行
程度とコンパクトであるにもかかわらず,他の IME に見られない機能を実装し
ており,本論文のような実験も容易である.

\subsection{MacRubyによる実装}

MacRuby は,Mac OS 用のアプリケーションを開発するために拡張された Ruby 実
行環境であり,Mac OS の Objective-C ライブラリを Ruby で扱うことが出来る.
%
{\system} では,Mac OS のIMEフレームワークである
InputMethodKit Framework を MacRuby から呼び出すことによって
基本的なIME の機能を実装している.

% \subsection{{\system}が受け取るキー入力}
\subsection{修飾キーの扱い}

エディタが編集機能のために用いる,ControlやCommandなどの修飾キーはIMEが
ハンドルしないのが普通であるが,{\system}では、一部のファンクションキー
と修飾キーの入力を受け取り,共通化されたテキスト編集操作を実現している.

{\system} のように,修飾キーの一部をハンドルする実装を行った場合,修飾
キーの入力はエディタに送出されないため,
%エディタが同じキー入力を編集操作に割り当てていても、機能衝突が起こることはない。
エディタが同じキー入力を編集操作の一部に割り当てている場合,それを利用することは出来なくなる.

\subsection{テキストデータの取得}

ブロック移動やインデントの処理といったテキスト編集の機能を,
あらゆるテキストエリアで行うためには,
テキストエリアに入力されている全文をIMEが取得する必要があり,
InputMethodKitはこの機能を備えていない.
%
{\system} では,
ブロック移動やインデントのコマンドが入力された際に,
Mac OS に実装されている,
テキストエリアの全文を選択状態にするコマンドを
AppleScriptにより送信している.

Mac OSでは,
多くのアプリケーションにおいてテキストエリアはNSTextField Class のオブジェクトとして実装されており,
このNSTextFieldに対してテキスト編集や入力の操作を行うためのNSTextInput Protocolが存在する.
このプロトコルには標準の実装としてCommand+Aキーによる全文選択の機能が備わっている.

{\system}では,
このキー入力をAppleScriptによりテキストエリアに送信することにより,
テキストエリアの全文を選択状態にしている.
%
{\system}では,
NSTextInputの標準実装であるCommand+Cキーによる全文コピーの機能をAppleScriptにより送信することで,
Mac OSのクリップボードにテキストエリアの全文を保存し,
それをInputMethodKit Frameworkが読み込むことで,
テキストエリアの全文をIMEが取得している.
%

\subsection{実装手法の問題点}

しかし,このような手法で実装できる編集機能や,
IMEとしてのパフォーマンスには問題がある.
たとえば,対象とするテキストエリア自体がNSTextField Classを利用していない場合や,
対象とするテキストエリアがテキストの全選択やコピーに割り当てるキーバインドを変更している場合,
テキストエリアの全文を取得することが出来ない場合があり,
{\system}の持つ編集機能は有効に機能しない.
%
また、どこまでの編集機能をエディタやテキストエリアから切り離し,
どれだけの拡張性を持たせることができるか,という点において,
IMEの層での実装には問題がある.
各国語入力のための枠組みであるIMEは,
テキストエリアよりも先にユーザのキー入力をハンドルすることができるが,
エディタが想定するキー入力を横取りして、エディタの機能をを打ち消す実装になる可能性がある.
%
前述したDynamic Macroの例のように、
編集操作をマクロ化して行うことは多くあるので,
IMEとテキストエリアとの間に,テキスト編集のための枠組みを用意し,
機能拡張が容易に可能なOSとAPI設計のレベルでの実装の修正が求められる.

