\section{議論}

\subsection{本手法の有用性}
パソコンやスマートフォンで利用されている様々なテキスト入力システムは変
換方式も使い勝手も全く異なっているのが普通になっているが、単純で柔軟な
入力方式を利用すると、パソコンでもスマートフォンでもほぼ共通の入力を行
なうことが可能である。{\system}はこのような思想にもとづいて作成された IME
であるが、本論文のような手法を取り入れることにより、あらゆる機器におい
て入力も編集もユニバーサルにすることが可能であろう。

\subsection{本手法の展望}
本論文ではテキストエディタに絞った説明を行なったが、IME はユーザ
のすべてのキー入力を直接受け取る窓口になっているため、編集と関係無いキー操作も IME に担当させることによってより幅広い計算機操作を実行することが
できる。たとえばシステム音量や画面の明るさをコントロールするにはシステ
ムに用意された特別のキーを使ったり、システムに用意されたショートカット
を利用したりすることが多いが、このようなものも IME から制御するようにし
ておけばシステム全体のショートカット設定などは不要になる。

\subsection{本手法の限界}

IME はアプリケーションと独立に実装されているため、
現在のような実装ではアプリケーションの内部状態によって動作を変えたり、
アプリケーションの振る舞いを制御したりすることはできず、表に出ているテキストの編集操作しか
できない。 既存のシステム自体は変更せずに、皮をかぶせる形で機能を拡張す
る手法はある程度有用ではあるが、問題の根本的な解決が必要な場合には限界
がある。テキスト編集の場合は根本的に解決しなければならない問題は多くな
いので、本論文の手法はとりあえず有効だといえるが、根本的な解決のために
は、テキスト入力の枠組みであるIMEに加えて、各コンピュータがテキスト編集
のための枠組みを用意する必要があるだろう。
