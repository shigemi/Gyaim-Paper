\section{はじめに}

計算機上でテキストを編集するために様々なテキストエディタが利用されている。
文書を作成するときはワープロを利用し、
メールを書くにはメールクライアントを利用し、
文字端末でのプログラム開発にはvimやEmacsを利用し、
IDEを利用した開発では付属のエディタを利用し、
ネット上でテキストを扱うにはブラウザのテキストフォームを利用するといったように、
場合に応じて様々なエディタが利用されている。

エディタの機能や操作体系はシステムごとに異なっているのが普通である。
ブラウザやワープロでテキストを1行消したい場合は
マウスで行全体を選択してから削除キーを押せばよいが、
vimでは「d」キーを2回タイプして消すことが多く、
EmacsではCtrl-Kキーがよく使われている。
Emacsに慣れたユーザがワープロ上でもCtrl-Kで行を消去したいと思っても、
そういう機能は用意されていないのが普通であるし、
機能拡張が可能なシステムを利用している場合でも、
操作体系を完全に同じにすることは難しい。
%
あらゆるエディタの操作を統一することは難しいが、
様々なエディタで共通に利用できるソフトウェア層を
ユーザとアプリケーションの間に置くことができれば
異なるエディタの編集操作をある程度共通化できる可能性がある。

現在のパソコンには IME(Input Method Editor) と呼ばれる文字入力機構が用意されており、
様々な言語のテキスト入力に利用されている。
IMEはエディタなどとは独立したソフトウェアであり、
ユーザのすべてのキー入力を受け取って
各国語に変換した結果をアプリケーションに送出する。
IMEはあらゆるアプリケーションで共通に利用されるので、
たとえば日本語入力用のIMEを利用すれば、
Emacs でもブラウザでも IDE でも同じ操作で日本語を入力できる。
IMEは一般には各国語入力のみのために利用されているが、
テキストの挿入/移動/削除といった編集操作もIMEが受け持つようにすれば、
様々なエディタ上で同じ操作で編集を行なうことが可能になる。
%
Mac上の様々なテキストエディタにおいて
同じキー操作でテキスト編集を可能にする{\system}システムについて述べる。

