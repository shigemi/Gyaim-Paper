\section{実装例}

MacRubyで記述された IME である「Gyaim」に様々な編集機能を追加したものを
利用した編集作業の例を示す。一般にIME はアルファベット以外の文字を入力
するときだけ有効にするのが普通であるが、ここでは Gyaim を 常に有効にし
ておくことによりあらゆるキー入力を Gyaim が取得している。

\subsection{ブロック移動}

%%% グラフの例
%\begin{figure*}[htb]
%\centering
%\includegraphics[scale=0.6]{graph.eps} 
%\caption{グラフの例}
%\label{figure:graph}
%\end{figure*} 
\subsection{文字列暗号化}
\subsection{Dynamic Macro}

以上の機能に加え、文字のスワップやキャレット移動といった基本的な編集操作をGyaimは実装しており、システムやアプリケーションによらないテキスト編集が可能である。
