\section{実装}
{\system} は MacRuby で記述された Mac 用の IME であり、ソースが 500 行程度とコンパクトであるにもかかわらず、他の IME に見られない機能を実装しており、本論文のような実験も容易である。

\subsection{MacRuby}
MacRuby は、Mac 用のアプリケーションを開発するために拡張された Ruby 実行環境であり、Mac の Objective-C ライブラリを Ruby で扱うことが出来る。
{\system} では、日本語などの入力を補助するフレームワークである InputMethodKit Framework を MacRuby から呼び出すことによって基本的な IME の機能を実装している。

\subsection{InputMethodKit Framework}
Mac OSX用に用意されたフレームワークであるInputMethodKit Frameworkは、全てのキー入力をハンドルする機能と、アクティブなアプリケーションに対してテキストを送出する機能を備えている。
一般的に、このような枠組みを利用して各国語用のIMEを開発する際には、エディタが編集機能のために用いる、ControlやCommandなどの修飾キーはハンドルしないのが普通であるが、{\system}では修飾キーに応じた編集操作を実現している。

しかし、ブロック移動やインデントの処理をあらゆる テキストエリアで行うためには、テキストフィールドに 入力されている全文をIMEが取得する必要があり、InputMethodKit はこの機能を備えていない。{\system} では、AppleScript などを併用することによりこの問題を解決したが、実装には課題が残っている。
