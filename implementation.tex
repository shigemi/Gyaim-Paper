\section{実装}

{\system} は MacRuby で記述された Mac 用の IME であり、ソースが 500 行
程度とコンパクトであるにもかかわらず、他の IME に見られない機能を実装し
ており、本論文のような実験も容易である。

\subsection{MacRubyによる実装}

MacRuby は、Mac 用のアプリケーションを開発するために拡張された Ruby 実
行環境であり、Mac の Objective-C ライブラリを Ruby で扱うことが出来る。
%
{\system} では、MacのIMEフレームワークである
InputMethodKit Framework を MacRuby から呼び出すことによって
基本的なIME の機能を実装している。

% \subsection{{\system}が受け取るキー入力}
\subsection{修飾キーの扱い}

エディタが編集機能のために用いる、ControlやCommandなどの修飾キーはIMEが
ハンドルしないのが普通であるが、{\system}では、一部のファンクションキー
と修飾キーの入力を受け取り、共通化されたテキスト編集操作を実現している。

{\system} のように、修飾キーの一部をハンドルする実装を行った場合、修飾
キーの入力はエディタに送出されないため、エディタが同じキー入力を編集操
作に割り当てていても、機能衝突が起こることはない。

\subsection{テキストデータの取得}

ブロック移動やインデントの処理をあらゆる テキストエリアで行うためには、
テキストフィールドに 入力されている全文をIMEが取得する必要があり、
InputMethodKit はこの機能を備えていない。
%
{\system} では、ブロック移動やインデントのコマンドが入力された際に、
Mac OSXに実装されている、テキストエリアの全文を選択状態にするコマンドを
AppleScriptにより送信している。これによりテキストエリアの全文を
InputMethodKitにより読み込み、全文を書き換えることで、テキスト編集の機
能を実現している。しかし、このような手法で実装できる編集機能には限界が
あり、課題が残っている。
